%%%%%%%%%%%%%%%%%%%%%%%%%%%%%%%%%%%%%%%%%%%%%%%%%%%%%%%%%%%%%%%%%%%%%%%%%%%%%%%%

% Nanny Contract
% ==============

% This is a contract defining the legal relationship between an Employer and an
% Employee hired as a nanny and domestic worker.

% Copyright (c) 2017 Sean Leather
% Licensed under CC BY 4.0. See the accompanying LICENSE.txt.

% Sources of inspiration for this contract:
%
% * Labour Guide Example Contract of Employement
%   http://www.labourguide.co.za/contracts-of-employments/228-contract-of-employment-example
%
% * Department of Labour Sample Employment Contract
%   http://www.labour.gov.za/DOL/documents/forms/basic-conditions-of-employment/sample-employment-contract/
%
% * Amended Basic Conditions of Employement Act (2002)
%   http://www.labour.gov.za/DOL/downloads/legislation/acts/basic-conditions-of-employment/Amended%20Act%20-%20Basic%20Conditions%20of%20Employment.pdf

%%%%%%%%%%%%%%%%%%%%%%%%%%%%%%%%%%%%%%%%%%%%%%%%%%%%%%%%%%%%%%%%%%%%%%%%%%%%%%%%

\documentclass[a4paper,11pt]{article}

%%%%%%%%%%%%%%%%%%%%%%%%%%%%%%%%%%%%%%%%%%%%%%%%%%%%%%%%%%%%%%%%%%%%%%%%%%%%%%%%

% Decrease margins
\usepackage[margin=2cm]{geometry}

%-------------------------------------------------------------------------------

% AMS math
\usepackage{amsmath}

% Support AMS symbols (e.g. \square)
\usepackage{amssymb}

% Use fonts with XeLaTeX
\usepackage{fontspec}

% Set the main serif and sans-serif fonts to the locally stored fonts.
% https://tex.stackexchange.com/a/12568/16670
\setmainfont
  [ Path = ./fonts/Charter/
  , Extension = .otf
  , UprightFont = *-Regular
  , ItalicFont = *-Italic
  , BoldFont = *-Bold
  , BoldItalicFont = *-BoldItalic
  ]{Charter}
\setsansfont
  [ Path = ./fonts/CooperHewitt/
  , Extension = .otf
  , UprightFont = *-Medium
  , ItalicFont = *-MediumItalic
  , BoldFont = *-Bold
  , BoldItalicFont = *-BoldItalic
  ]{CooperHewitt}

% Set the fonts for the section headers
\usepackage{titlesec}
\titleformat*{\section}{\Large\bfseries\sffamily}
\titleformat*{\subsection}{\large\sffamily}
\titleformat*{\subsubsection}{\itshape\sffamily}

%-------------------------------------------------------------------------------

% Number all the section headers down to and including \paragraph
% Useful: https://tex.stackexchange.com/q/305220/16670
\setcounter{secnumdepth}{4}

% Define a command to start a paragraph with no header text
\NewDocumentCommand{\para}{}{\paragraph{}}

%-------------------------------------------------------------------------------

% Use 'tabularx' tables.
% https://en.wikibooks.org/wiki/LaTeX/Tables#The_tabularx_package
\usepackage{tabularx}

% Ragged column types for 'tabularx':
% Align to one side and fill out the available space.
\newcolumntype{L}{>{\raggedright\arraybackslash}X}% Left align (space right)
\newcolumntype{R}{>{\raggedleft\arraybackslash}X}%  Right align (space left)

%-------------------------------------------------------------------------------

% Define a command for an empty underline to be used in a form. The height of
% the line is hardcoded, and the width is provided in an argument.
\NewDocumentCommand{\Line}{m}{
  % From https://tex.stackexchange.com/a/179192/16670
  \rule{#1}{0.15mm}
}

%-------------------------------------------------------------------------------

% Page number footer (current of last)
% Adapted from:
% http://latex.org/forum/viewtopic.php?t=6835
% https://tex.stackexchange.com/q/227/16670
\usepackage{fancyhdr}
\usepackage{lastpage}
\pagestyle{fancy} % Customize the header and footer
\fancyhf{} % Clear header and footer
\renewcommand\headrulewidth{0pt} % Remove the header rule
% \fancyfoot[C]{\footnotesize{}Page \thepage\ of \pageref{LastPage}} % Footer
\lfoot{\footnotesize{}Page \thepage\ of \pageref{LastPage}} % Left footer
\rfoot{\footnotesize{}Employee's initials: \Line{1cm}} % Right footer

\begin{document} %%%%%%%%%%%%%%%%%%%%%%%%%%%%%%%%%%%%%%%%%%%%%%%%%%%%%%%%%%%%%%%

% Left-justify entire document
\raggedright

\begin{center}
  {\Huge \textsf{Contract of Employment}}

  \vspace{2cm}

  \textbf{Entered into between:}

  \vspace{0.5cm}

  \Line{0.75\textwidth}

  (\textit{herein after referred to as ``the Employer''})

  \vspace{0.5cm}

  \Line{0.75\textwidth}

  \Line{0.75\textwidth}

  (\textit{Employer address})

  \vspace{0.5cm}

  \Line{0.75\textwidth}

  (\textit{Employer telephone})

  \vspace{0.5cm}

  \textbf{and:}

  \vspace{0.5cm}

  \Line{0.75\textwidth}

  (\textit{herein after referred to as ``the Employee''}).

  \vspace{0.5cm}

  \Line{0.75\textwidth}

  \Line{0.75\textwidth}

  (\textit{Employee address})

  \vspace{0.5cm}

  \Line{0.75\textwidth}

  (\textit{Employee telephone})

  \vspace{0.5cm}

  \Line{0.75\textwidth}

  (\textit{Employee South Africa ID number or passport number and country of
  residence})

  \vspace{0.5cm}

\end{center}

\noindent\hrulefill %-----------------------------------------------------------

\section{Terms and conditions of employment}

\para The terms and conditions set out herein will constitute the contract
between the Employee and the Employer. Where a basic condition of employment is
not specifically mentioned, the relevant legislation will be applicable (e.g.\
the Basic Conditions of Employment Act, No. 75 of 1997, the Labour Relations
Act, No. 66 of 1995, amendments to legislation, etc.).

\section{Commencement of employment}
\label{commencement}

\para This contract will take effect from the date:

\begin{center}
  \Line{0.75\textwidth}

  (\textit{day-month-year})
\end{center}

\noindent and continue until terminated as in Section~\ref{termination}.

\section{Probation}
\label{probation}

\para The Employee is initially employed for a probation period of 3 calendar
months from the date of commencement given in Section~\ref{commencement}. The
purpose of probation is to determine the suitability of the Employee for the
position.

\para The probation period ends on the date:

\begin{center}
  \Line{0.75\textwidth}

  (\textit{day-month-year})
\end{center}

\section{Termination of employment}
\label{termination}

\para Either the Employer or the Employee may terminate the agreement with
written notice. In case the Employee is illiterate, the notice may be given by
the Employee verbally.

\para This contract of employment may be terminated only:

\begin{enumerate}

  \item on \textit{immediate} notice if the Employee has been employed for 3
    calendar months or less (as given in Section~\ref{probation});

  \item on notice of \textit{not less than 1 week} if the Employee has been
    employed for 6 calendar months or less; or

  \item on notice of \textit{not less than 4 weeks} if the Employee has been
    employed for more than 6 calendar months.

\end{enumerate}

\section{Place of work}

\para Place of work:

\begin{center}
  \Line{0.75\textwidth}

  \Line{0.75\textwidth}

  (\textit{address})
\end{center}

\section{Job description}

\subsection{Job title}
\label{job-title}

\para Job title:

\begin{center}
   \textbf{Child minder (nanny) and domestic worker}
\end{center}

\subsection{Duties}
\label{duties}

\para The main duties of this position are listed in the \textit{Job duties}
found in Appendix~\ref{job-duties}.

\para The Employee may occasionally be asked to perform other duties that would
reasonably be expected given the job title in Section~\ref{job-title}.

\section{Wages and benefits}
\label{wages}

\paragraph{Monthly wage}\label{monthly-wage} The Employee's total monthly wage
will be:

\begin{center}
  R \Line{0.75\textwidth}
\end{center}

\paragraph{Hourly wage}\label{hourly-wage} This is equivalent to an hourly wage
(as in Section 35 (Calculation of remuneration and wages) of the Basic
Conditions of Employment Act, No. 75 of 1997, as amended by the Basic
Conditions of Employment Act, No. 11 of 2002) of:

\begin{center}
  R \Line{0.75\textwidth}

  (\((\text{\textit{value of Paragraph}~\ref{monthly-wage}}) * 3 / 13 / 42.5 \))
\end{center}

\para\label{monthly-wage-pay-day} The monthly wage will be paid on the last
working day of every month.

\para The Employer will review and may revise the wages once a year from the
date of commencement given in Section~\ref{commencement}.

\paragraph{Meals} The Employee is entitled to 1 meal for every full working
day.

\paragraph{Clothing} The Employee is entitled to 1 uniform for every 36 calendar
months of continuous service.

\paragraph{Overtime}\label{overtime-wage} For overtime, the Employee will be
paid:

\begin{enumerate}

  \item one and a half (1.5) times the normal hourly wage, as in\par
    Paragraph~\ref{hourly-wage}; and

  \item at the same time as other wages, as in
    Paragraph~\ref{monthly-wage-pay-day}.

\end{enumerate}

\section{Working time}
\label{working-time}

\para Normal working hours will be:

\begin{center}
   \textbf{7:30 to 16:30}

   \textbf{Monday to Friday}
\end{center}

\para The Employee is entitled to a lunch break of 30 minutes during normal
working hours. This means that a typical week of work is considered to be 42.5
hours.

\para The Employee will not be required to work on Saturdays, Sundays, or
public holidays unless agreed to by the Employee and Employer. If agreed, the
Employee will receive overtime wages, as in Paragraph~\ref{overtime-wage}, for
the hours worked on such days.

\para Hours worked outside of the normal working hours on days other than
Saturdays, Sundays, and public holidays are also subject to overtime wages.

\para Overtime will only be worked from time to time if agreed upon between the
Employer and Employee.

\section{Leave}
\label{leave}

\para The Employee is entitled to paid leave only as specified in
Sections~\ref{annual-leave}, \ref{sick-leave},
and~\ref{family-responsibility-leave}.

\para Unpaid leave is specified in Section~\ref{unpaid-leave}

\subsection{Annual leave}
\label{annual-leave}

\para The Employee is entitled to 24 days paid annual leave during every 12
calendar months of continuous employment.

\para Annual leave must be taken at times agreed to by the Employer and the
Employee to allow for the Employer to arrange care for the child, if necessary.

\para The Employer may require the Employee to take leave at times that
coincide with the leave of the Employer.

\para Annual leave does not accumulate from year to year.

\para For annual leave, the Employee will be paid:

\begin{enumerate}

  \item normal wages, as in Paragraph~\ref{monthly-wage}; and

  \item at the same time as other wages, as in
    Paragraph~\ref{monthly-wage-pay-day}.

\end{enumerate}

\subsection{Sick leave}
\label{sick-leave}

\para The Employee is entitled to 10 days of paid sick leave during every 12
calendar months of continuous employment.

\para The Employee must notify the Employer as soon as possible in the case of
an absence from work for illness.

\para A medical certificate is required if the Employee is absent for more than
2 consecutive days or is absent on more than 2 occasions during an 8-week
period. If no medical certificate is provided on request, the Employer is not
required to pay the Employee.

\para Sick leave does not accumulate from year to year.

\para For sick leave, the Employee will be paid:

\begin{enumerate}

  \item normal wages, as in Paragraph~\ref{monthly-wage}; and

  \item at the same time as other wages, as in
    Paragraph~\ref{monthly-wage-pay-day}.

\end{enumerate}

\subsection{Family responsibility leave}
\label{family-responsibility-leave}

\para The Employee is entitled to 3 days paid family responsibility leave during
every 12 calendar months of continuous employment.

\para Family responsibility leave may be taken:

\begin{enumerate}

  \item when the Employee's child is born;

  \item when the Employee's child is sick; or

  \item in the event of the death of the Employee's spouse, parent, grandparent,
    grandchild, or sibling.

\end{enumerate}

\para The Employee must notify the Employer as soon as possible in case of an
absence from work for family responsibility.

\para Family responsibility leave does not accumulate from year to year.

\para For family responsibility leave, the Employee will be paid:

\begin{enumerate}

  \item normal wages, as in Paragraph~\ref{monthly-wage}; and

  \item at the same time as other wages, as in
    Paragraph~\ref{monthly-wage-pay-day}.

\end{enumerate}

\subsection{Unpaid leave}
\label{unpaid-leave}

\para Unpaid leave may be granted to the Employee when:

\begin{enumerate}

  \item annual leave has been depleted;

  \item sick leave cannot be reasonably substantiated or has been depleted; and

  \item family responsibility leave cannot be reasonably substantiated or has
    been depleted.

\end{enumerate}

\para The Employee may apply in writing to be granted unpaid leave for an
extended period not exceeding 1 month in exceptional circumstances. The
application must be fully motivated. The Employer is not obliged to approve
such an application.

\para If the Employee is absent from duty without prior arrangement or
permission, the Employer may regard any period of such absence as unpaid leave.
This does not preclude the Employer from taking disciplinary measures against
the Employee in terms of this contract.

\para For unpaid leave, the Employee is not entitled to wages.

\section{Changes to contract}
\label{changes}

\para Any changes to this contract will only be valid if they are in writing
and have been agreed upon and signed by both parties.

\pagebreak %--------------------------------------------------------------------

\rfoot{}

\begin{center}
  Thus done and signed at:

  \vspace{0.5cm}

  \Line{0.75\textwidth}

  (location)

  \vspace{0.5cm}

  on:

  \vspace{0.5cm}

  \Line{0.75\textwidth}

  (\textit{day-month-year})

  \vspace{0.5cm}

  by:

  \vspace{0.5cm}

  \Line{0.75\textwidth}

  (\textit{signature of the Employer})

  \vspace{0.5cm}

  and:

  \vspace{0.5cm}

  \Line{0.75\textwidth}

  (\textit{signature of the Employee})

  \vspace{0.5cm}

  with witnesses (names and signatures):

  \vspace{0.5cm}

  \Line{0.75\textwidth}

  \vspace{0.5cm}

  \Line{0.75\textwidth}

\end{center}

\pagebreak %--------------------------------------------------------------------

\appendix

\section{Job duties}
\label{job-duties}

This appendix describes the main job duties as referenced in
Section~\ref{duties}.

\begin{enumerate}
  \item Minding a child
    \begin{itemize}
      \item Ensuring the safety of the child at all times
      \item Feeding breast milk and formula from bottles
      \item Feeding solids
      \item Changing nappies and wiping the child
      \item Exercising and entertaining the child
    \end{itemize}
  \item Cleaning house
    \begin{itemize}
      \item Sweeping floors
      \item Mopping floors
      \item Dusting
      \item Wiping appliances (e.g.\ TV, etc.)
      \item Cleaning walls, light switches, doors, etc.
      \item Cleaning kitchen counters
      \item Cleaning toilets, basins, baths, showers, taps, etc.
      \item Cleaning cupboards
      \item Cleaning stoves or ovens
      \item Cleaning refridgerator and freezer
      \item Cleaning windows and glass doors
      \item Removing refuse for collection
    \end{itemize}
  \item Laundry
    \begin{itemize}
      \item Hand washing
      \item Hanging laundry to dry
      \item Folding laundry
      \item Washing curtains
      \item Ironing
    \end{itemize}
\end{enumerate}

\end{document} %%%%%%%%%%%%%%%%%%%%%%%%%%%%%%%%%%%%%%%%%%%%%%%%%%%%%%%%%%%%%%%%%
